\documentclass[11pt, a4paper]{article}
\usepackage{mathtools}
\usepackage{caption}
\usepackage{float}
\usepackage[slovene]{babel}
\usepackage{amssymb}
\usepackage{amsthm}
\usepackage{enumitem}



\hyphenpenalty=10000

\begin{document}
    \newtheorem{theorem}{Izrek}[section]
    \newtheorem{definition}[theorem]{Definicija}
    \newtheorem{corollary}[theorem]{Posledica}
    \newtheorem{lemma}[theorem]{Lema}
    \newtheorem*{remark}{Opomba}
    \newtheorem{poroposition}[theorem]{Trditev}
    \newtheorem{example}[theorem]{Zgled}


    \title{Optimizacijske metode}
    \author{Napisal Jure Pustoslemšek po zapiskih predavanj izr. prof. dr. Matjaža Konvalinke}
    \date{Junij 2020}
    \maketitle

    \section{Optimizacijski problemi}

    Hocemo maksimizirati ali minimizirati realno funkcijo, definirano na neki mnozici.

        \begin{example}
            Minimum/maksimum funkcije \(f(x)=x^2-2x+4\) na \([-2,2]\)
            \par
            Uporabimo odvod funkcije.
        \end{example}

        \begin{example}
            Minimum/maksimum funkcije \(f(x,y)=x^2-y^2\) na \([-3,1]\times[0,2]\)
            \par
            Gledamo parcialne odvode in vrednosti funkcije na robu.
        \end{example}

        \begin{example}[Problem kmetije]
            \par
            Na kmetiji s \(50 ha\) pridelovalne površine želimo maksimizirati dobiček po spodnji tabeli

            \begin{center}
                \begin{tabular}{ c|c|c|c }
                    pridelek & ure dela & stroski & dobiček \\
                    \hline
                    pšenica & 60 & 400 & 240 \\
                    koruza & 80 & 600 & 400 \\
                    krompir & 100 & 480 & 320 \\
                \end{tabular}
            \end{center}
            
            Na voljo imamo 5.000 ur delovne sile in 24.000€ kapitala.
            
            \par
            Kaj nas zanima? Zanima nas, kako bomo razporedili različne pridelke po pridelovalni površini za najvecji dobiček.

            \par
            \(x_1\) . . . površina pšenice v \(ha\)
            \newline
            \(x_2\) . . . površina koruze v \(ha\)
            \newline
            \(x_3\) . . . površina krompirja v \(ha\)
            \newline
            
            Problem kmetije lahko matematično izrazimo takole: \\

            \begin{center}
                \begin{tabular}{ cc }
                    $max$ & \(
                        240x_1+400x_2+320x_3
                    \) \\
                    $p.p.$ & \(
                        x_1+x_2+x_3 \le 50
                    \) \\
                    & \(
                        60x_1+80x_2+100x_3 \le 5.000
                    \) \\
                    & \(
                        400x_1+600x_2+46x_3 \le 24.000
                    \) \\
                    & \(
                        x_1,x_2,x_3 \ge 0    
                    \) \\
                \end{tabular}
            \end{center}

            To je \emph{linearni program (LP)}.
        \end{example}

        \begin{example}
            Imamo \(2n\) jabolk, ki tehtajo \(w_1,...,w_{2n}\)
            \par
            Jabolka razvrstimo v dve košari tako, da je v vsaki \(n\) jabolk, in da sta teži kosar cim bolj podobni.

            \par
            Jabolka v 1. košari ponazorimo z množico A. Ker je v množici A natanko polovica jabolk, lahko 2. košaro ponazorimo z množico \(A^C\).

            \par
            Torej iščemo
            \begin{center}
                \[
                    min |\sum_{w_i \in A}{w_i} - \sum_{w_i \in A^C}{w_i}|
                \]
            \end{center}
            
            \par
            Poskusimo ta problem predstaviti kot LP: \\
            \(x_i\); \(i \in [2n]\) \\
            \(x_i = 1\), če damo \(i\)-to jabolko v levo košaro. \\
            \(x_i = -1\), če damo \(i\)-to jabolko v desno košaro. \\

            \begin{table}[h!]
                \[
                    \begin{array}{ cc }
                        min &
                            \displaystyle |\sum_{i=1}^{2n}{w_i x_i}| \\
                        
                        p.p. & 
                            \displaystyle \sum_{i=1}^{2n}{x_i} = 0 \\
                        &
                            x_i \in \{-1, 1\} \\
                    \end{array}
                \]
            \end{table}
            
            To \underline{ni} linearen program! Absolutna vrednost ni linearna.
        \end{example}

    \begin{definition}[Optimizacijski problem]
        Optimizacijski problem karakteriziramo kot \((min/max, f, \Omega)\), kjer je \(f\) kriterijska funkcija (tj. funkcija, ki jo zelimo minimizirati oz. maksimizirati), \(\Omega\) pa množica dopustnih rešitev, tj. vseh \(x\), ki ustrezajo danim pogojem. 
    \end{definition}
    Zanima nas \underline{optimalna vrednost}, tj. \(x^* \in \Omega\), za katerega je \(f(x^*)\) največji oz. najmanjši mozen.

    \par
    Optimizacijski problem je:
    \begin{itemize}
        \item Nedopusten, ce \(\Omega = \emptyset\),
        \item Dopusten, ce je \(\Omega \neq \emptyset\)
        
    \end{itemize}

    \par
    Dopustne probleme delimo na:
    \begin{itemize}
        \item Neomejen, ce je \(\Omega\) neomejena, torej lahko vedno najdemo boljšo rešitev,
        \item Omejen, ce je \(\Omega\) omejena
        
    \end{itemize}

    \par
    Omejene probleme pa še delimo na:
    \begin{itemize}
        \item Optimalen, če obstaja optimalna rešitev,
        \item Neoptimalen, če lahko vedno najdemo boljšo rešitev
    \end{itemize}



    \section{Linearno programiranje}

    \begin{definition}[Linearni program]
        Linearni program (LP) je optimizacijski problem, v katerem je kriterijska funkcija linearna, dopustna množica pa je podana z linearnimi enačbami in neenačbami.
    \end{definition}

    \begin{example}
        \[
            \begin{array}{ c c }
                min & 2x_1-3x_2+x_3 \\
                p.p. & x_1+x_2 \le 4 \\
                & 3x_1-x_2-x_3 \ge 1 \\
                & 2x_2-5x_3 = -1 \\
                & x_1 \ge 0 \\
                & x_3 \le 0 \\
            \end{array}
        \]
    \end{example}
    

    \begin{definition}[Standardna oblika LP]
        LP je v \underline{standarni obliki}, ce:
        \begin{enumerate}
            \item je tipa \(max\),
            \item vse omejitve so oblike \(g_j(x) \le b_j\) in
            \item za vse spremenljivke so omejene z \(x_i \ge 0\)
        \end{enumerate}
    \end{definition}

    \begin{figure}[h!]
        \[
            \begin{array}{ c c }
                max & c_1x_1+...+c_nx_n \\
                p.p. & a_{11}x_1+...+a_{1n}x_n \le b_1 \\
                & a_{21}x_1+...+a_{2n}x_n \le b_2 \\
                & . \\
                & . \\
                & . \\
                & a_{m1}x_1+...+a_{mn}x_n \le b_m \\
                & x_1,x_2,...,x_n \le 0 \\
                
            \end{array}
        \]
        \caption{LP v standardni obliki}
    \end{figure}

    \par
    LP v standardni obliki lahko izrazimo tudi v matričnem zapisu.  
    
    \begin{figure}[H]
        \[
            \begin{array}{ c c c c }
                x = 
                \begin{bmatrix}
                    x_1 \\
                    x_2 \\
                    . \\
                    . \\
                    x_n    
                \end{bmatrix}
                &
                c =
                \begin{bmatrix}
                    c_1 \\
                    c_2 \\
                    . \\
                    . \\
                    c_n
                \end{bmatrix}
                &
                b = 
                \begin{bmatrix}
                    b_1 \\
                    b_2 \\
                    . \\
                    . \\
                    . \\
                    b_m
                \end{bmatrix}
                &
                A = 
                \begin{bmatrix}
                    a_{ij}    
                \end{bmatrix}
                \in \mathbb{R}^{m \times n}
            \end{array}
        \]
    \end{figure}
    
    \par
    S temi oznakami lahko vsak LP (tudi takšne, ki niso v matrični obliki) zapišemo v matričnem zapisu:
    
    \begin{figure}[H]
        \[
            \begin{array}{ c c }
                max & c^Tx \\
                p.p. & Ax \le b \\
                & x \ge 0 \\
            \end{array}
        \]
        \caption{LP v standardni obliki (matrični zapis)}
    \end{figure}
    
    \begin{theorem} 
        Vsak LP lahko pretvorimo v ekvivalentni LP v standardni obliki.
    \end{theorem}

    \begin{proof}
        Pretvorbo poljubnega LP v standardno obliko opravimo v več korakih:
        \begin{enumerate}
            \item Minimizacijski problem je ekvivalenten maksimizacijskemu problemu negativne kriterijske funkcije z negativnimi pogoji.
            \[min f(x) \rightarrow max -f(x)\]
            \[a_{ij}x_i = b_j \rightarrow -a_{ij}x_i = -b_j\]
            \[a_{ij}x_i \le b_j \rightarrow -a_{ij}x_i \le -b_j\]
            \[a_{ij}x_i \ge b_j \rightarrow -a_{ij}x_i \ge -b_j\]

            \item Omejitve spremenljivk popravimo z vpeljavo novih spremenljvk:
            \begin{enumerate}[label*=\arabic*.]
                \item \(x_i \ge b_j\) zamenjamo z novo spremenljivko \(x_i'=x_i-b_j\), da dobimo omejitev \(x_i' \ge 0\)
                
                \item \(x_i \le b_j\) zamenjamo z novo spremenljivko \(x_i'=b_j-x_i\), da dobimo omejitev \(x_i' \ge 0\)
                
                \item Neomejene spremenljivke zamenjamo z dvema novima spremenljivkama, podanima z enačbo \(x_i=x_i'-x_i''\), novi spremenljivki pa sta omejeni z omejitvijo \(x_i',x_i'' \ge 0\) 
            \end{enumerate}

            \item Vsak pogoj oblike \(g_j(x) = b_j\) pretvorimo v dva pogoja:
            \par
            \(g_j(x) \le b_j\) in \(g_j(x) \ge b_j\)

            \item Pogoje oblike \(g_j(x) \ge b_j\) pomnožimo z \(-1\). S tem ga pretvorimo v \(-g_j(x) \le -b_j\)
        \end{enumerate}
    \end{proof}

    \paragraph{TODO: Konveksne množice}
    \clearpage


    \subsection{Simpleksna metoda}

    Simpleksna metoda zahteva LP v standardni obliki:
    \begin{figure}[h!]
        \[
            \begin{array}{ c c }
                max & c^Tx \\
                p.p. & Ax \le b \\
                & x \ge 0 \\
            \end{array}
        \]
        \caption{LP v standardni obliki (matrični zapis)}
    \end{figure}

    Označimo z \(n\) število spremenljivk in z \(m\) število neenakosti, tj. \(c,x \in \mathbb{R}^n \), \(b \in \mathbb{R}^m\) in \(A \in \mathbb{R}^{m \times n}\).


    \begin{example}[Problem kmetije]
        \begin{figure}[h!]
            \[
                \begin{array}{ c c }
                    max & 240x_1+400x_2+320x_3 \\
                    p.p. & x_1+x_2+x_3 \le 50 \\
                    & 60x_1+80x_2+100x_3 \le 5000 \\
                    & 400x_1+600x_2+480x_3 \le 2400 \\
                    & x_1,x_2,x_3 \ge 0 \\
                \end{array}
            \]
            \caption{LP problema kmetije v standardni obliki}
        \end{figure}

        Koeficiente v kriterijski funkciji in vsaki omejitvi posebej bomo delili z največjim skupnim deliteljem, da bomo dobili ekvivalenten LP z "lepšimi" številkami.

        \begin{figure}[h!]
            \[
                \begin{array}{ c c }
                    max & 3x_1+5x_2+4x_3 \\
                    p.p. & x_1+x_2+x_3 \le 50 \\
                    & 3x_1+4x_2+5x_3 \le 250 \\
                    & 10x_1+15x_2+12x_3 \le 600 \\
                    & x_1,x_2,x_3 \ge 0 \\
                \end{array}
            \]
            \caption{LP problema kmetije z "lepšimi" koeficienti}
        \end{figure}

        Vsako neenakost spremenimo v enakost z novo spremenljivko.

        \begin{figure}[h!]
            \[
                \begin{array}{ c }
                    x_4=50-x_1-x_2-x_3 \\
                    x_5=250-3x_1-4x_2-5x_3 \\
                    x_6=600-10x_1-15x_2-12x_3 \\
                    \hline
                    z=3x_1+5x_2+4x_3 \\
                \end{array}
            \]
            \caption{Prvi slovar simpleksne metode na problemu kmetije}
        \end{figure}

        To je \underline{prvi slovar} simpleksne metode.
    \end{example}
    

    V splošnem ima slovar \(m+1\) linearnih enačb in \(n+m+1\) spremenljivk.
    \(m\) spremenljivk izmed \(x_1,x_2,...,x_{n+m}\) in \(z\) je izraženih z ostalimi \(n\) spremenljivkami.

    \par
    Teh \(m\) spremenljivk, ki so izražene z ostalimi \(n\) spremenljivkami, imenujemo \underline{bazne spremenljivke}, ostale spremenljivke pa so \underline{nebazne} (spremenljivka \(z\) predstavlja vrednost kriterijske funkcije in ne spada med bazne ali nebazne spremenljivke).

    \par
    \(x_1,x_2,...,x_n\) so \underline{prvotne} spremenljivke.
    \par
    \(x_{n+1},x_{n+2}...,x_{n+m}\) so \underline{dodatne} spremenljivke.

    \par
    V prvem slovarju so bazne spremenljivke natanko dodatne spremenljivke, nebazne pa prvotne.

    \par
    Rečemo, da je slovar dopusten, če so vsi konstatni koeficienti baznih spremenljivk nenegativni. Prvi slovar je torej dopusten natanko takrat, ko je \(b \ge 0\). Če je \(b \ngeq 0\), uporabimo \textit{dvofazno simpleksno metodo} (kasneje). Zato zaenkrat predpostavimo \(b \ge 0\). V tem primeru je LP zagotovo dopusten, saj je \(x_1=x_2=...=x_n=0\) dopustna rešitev.
    \par
    \begin{remark}
        Vse spremenljivke (tako prvotne kot dodatne) so nenegativne.
    \end{remark}
    
    \par
    Z dopustnim slovarjem imamo \textit{bazno dopustno rešitev}: nebazne spremenljivke nastavimo na 0. \(f(x)\) je vrednost kriterijske funkcije, \(x\) pa so vrednosti prvotnih spremenljivk (nebazne smo nastavili na 0, bazne pa imajo vrednosti konstatnih koeficientov).
    
    \paragraph{TODO: opis algoritma simpleksne metode}


    \begin{theorem}[Osnovni izrek linearnega programiranja]
        \begin{enumerate}
            \item Če ima LP dopustno rešitev, ima tudi bazno dopustno rešitev.
            \item Če ima LP optimalno rešitev, ima tudi bazno optimalno rešitev.
            \item Velja natanko ena od možnosti za dani LP:
            \begin{itemize}
                \item LP je nedopusten.
                \item LP je neomejen.
                \item LP je optimalen.
            \end{itemize}
        \end{enumerate}
    \end{theorem}
    \clearpage



    \subsection{Dualnost pri linearnem programiranju}

    \textbf{Ideja:} Problem kmetije \(P\)
    \begin{figure}[h!]
        \[
            \begin{array}{ c c }
                max & 3x_1+5x_2+4x_3 \\
                p.p. & x_1+x_2+x_3 \le 50 \\
                & 3x_1+4x_2+5x_3 \le 250 \\
                & 10x_1+15x_2+12x_3 \le 600 \\
                & x_1,x_2,x_3 \ge 0 \\
            \end{array}
        \]
        \caption{LP problema kmetije}
    \end{figure}

    Zanimajo nas zgornje meje za kriterijsko funkcijo. 
    \par
    Neenakosti pomnožimo z \(y_2,y_2,...,y_m \ge 0\) in seštejemo:
    \[(x_1+x_2+x_3)y_1+(3x_1+4x_2+5x_3)y_2+(10x_1+15x_2+12x_3)y_3 \le 50y_1+250y_2+600y_3\]
    \[(y_1+3y_2+10y_3)x_1+(y_1+4y_2+15y_3)x_2+(y_2+5y_2+12y_3)x_3 \le 50y_1+250y_2+600y_3\]
    
    Če velja:
    
    \[
        \begin{array}{c}
            y_1+3y_2+10y_3 \ge 3 \\
            y_1+4y_2+15y_3 \ge 5 \\
            y_1+5y_2+12y_3 \ge 4 \\ 
        \end{array}
    \]
    je \(2x_1+5x_2+4x_3 \le 50y_1+250y_2+600y_3\).

    Dualni problem kmetije \(P'\):
    \begin{figure}[h!]
        \[
            \begin{array}{ c c }
                min & 50 y_1+250y_2+600y_3 \\
                p.p. & y_1+3y_2+10y_3 \ge 3 \\
                & y_1+4y_2+15y_3 \ge 5 \\
                & y_1+5y_2+12y_3 \ge 4 \\
                & y_1,y_2,y_3 \ge 0 \\
            \end{array}
        \]
        \caption{Dualni problem kmetije}
    \end{figure}

    V splošnem:
    \begin{definition}[Dualni LP]
        Za dani LP P v standardni obliki:
        \[
            \begin{array}{ c c }
                max & c^Tx \\
                p.p. & Ax \le b \\
                & x \ge 0 \\
            \end{array}    
        \]
        je \textbf{dualni problem} LP \(P'\)
        \[
            \begin{array}{ c c }
                min & b^Ty \\
                p.p. & A^Ty \ge c \\
                & y \ge 0 \\
            \end{array}
        \]
    \end{definition}

    \begin{theorem}
        Za poljuben LP \(P\) je \(P''=P\).
    \end{theorem}

    \begin{proof}
        Naj bo LP \(P\) v standardni obliki. Potem je  \(P'\) v standardni obliki
        \[
            \begin{array}{ c c }
                max & -b^Ty \\
                p.p. & (-A^T)y \le -c \\
                & y \ge 0 \\
            \end{array}
        \]
        \(P''\) je potem:
        \[
            \begin{array}{ c c }
                min & -c^Tx \\
                p.p. & (-A^T)^Tx \ge -b \\
                & x \ge 0 \\
            \end{array}
        \]
        kar po pretvorbi v standardno obliko postane enako kot P:
        \[
            \begin{array}{ c c }
                max & c^Tx \\
                p.p. & Ax \le b \\
                & x \ge 0 \\
            \end{array}
        \]
    \end{proof}

    \begin{remark}
        Če ima LP \(P\) \(n\) spremenljivk in \(m\) neenačb, ima njegov dualni LP \(P'\) \(m\) spremenljivk in \(n\) neenačb.
    \end{remark}

    \begin{theorem}[Šibki izrek o dualnosti (ŠID)]
        Naj bo \(P\) linearni program, \(P'\) njegov dualni LP ter \(x\) in \(y\) dopustni rešitvi za \(P\) in \(P'\). Ob uporabi oznak od prej je potem \(c^Tx \le b^Ty\).
    \end{theorem}

    \begin{proof}
        Brez izgube splošnosti predpostavimo, da je LP \(P\) v standardni obliki in opazimo, da je \(A^Ty \ge c\) in \(Ax \le b\). Ker velja tudi \(x \ge 0\) in \(y \ge 0\):
        \[c^Tx \le (A^Ty)^T = y^TAx \le y^Tb = b^Ty\]
    \end{proof}

    \begin{corollary}
        Če velja \(c^Tx* = b^Ty*\) za dopustni rešitvi \(x*,y*\), je \(x*\) optimalna rešitev za \(P\) in \(y*\) optimalna rešitev za \(P'\).
    \end{corollary}

    \begin{proof}
        
    \end{proof}
\end{document}